\documentclass[11pt, oneside]{article}   	% use "amsart" instead of "article" for AMSLaTeX format
\usepackage{geometry}                		% See geometry.pdf to learn the layout options. There are lots.
\geometry{letterpaper}                   		% ... or a4paper or a5paper or ... 
%\geometry{landscape}                		% Activate for rotated page geometry
%\usepackage[parfill]{parskip}    		% Activate to begin paragraphs with an empty line rather than an indent
\usepackage{graphicx}				% Use pdf, png, jpg, or eps§ with pdflatex; use eps in DVI mode
								% TeX will automatically convert eps --> pdf in pdflatex		
\usepackage{amssymb}
\usepackage{upquote}

%-----------------------------------------------------------------------------
% Special-purpose color definitions (dark enough to print OK in black and white)
\usepackage{color}
% A few colors to replace the defaults for certain link types
\definecolor{orange}{cmyk}{0,0.4,0.8,0.2}
\definecolor{darkorange}{rgb}{.71,0.21,0.01}
\definecolor{darkgreen}{rgb}{.12,.54,.11}
%-----------------------------------------------------------------------------
% The hyperref package gives us a pdf with properly built
% internal navigation ('pdf bookmarks' for the table of contents,
% internal cross-reference links, web links for URLs, etc.)
\usepackage{hyperref}
\hypersetup{pdftex, % needed for pdflatex
  breaklinks=true, % so long urls are correctly broken across lines
  colorlinks=true,
  urlcolor=blue,
  linkcolor=darkorange,
  citecolor=darkgreen,
}

\usepackage{booktabs}


\title{Stat 222: Mouse Lifestyles}
%\date{}							% Activate to display a given date or no date

\begin{document}
\maketitle

\section{Data Description}

For this project, your primary data source will be mouse behavioral data from
the Tecott Lab at
UCSF.\footnote{\url{http://www.neuroscience.ucsf.edu/neurograd/faculty/tecott.html}}
The lab has recently developed a method for continuous high-resolution
behavioral data collection and analysis, which enables them to observe and
study the structure of spontaneous patterns of behavior (``Lifestyles'') in the
mouse \cite{tecott2003genes, tecott2004neurobehavioral, goulding2008robust,
anderson2014toward}.  They have found that using this method: 1) reveals a set
of fundamental principles of behavioral organization that have not been
previously reported, 2) permits classification by genotype with unprecedented
accuracy, and 3) enables fine dissection of behavioral patterns.

A central goal for this project is to give you hands-on experience working on a
real-world applied statistics\footnote{See Philip Stark's ``Thoughts on applied
Statistics'' here: \url{http://www.stat.berkeley.edu/~stark/other.htm}}
project.  In the real-world, problems are not pre-packaged as a numbered list
of short questions.  So understanding the problem is often the first step in an
applied statistics project.  As you start to come to an understanding of the
problem, questions will emerge.  And as the questions become clear, you will
need to determine how you will attempt to answer them.  This process will be
slow and at times vexing.  While the Tecott Lab, Johnny, and I will provide
guidance on the project, you will ultimately be responsible for determining
what you do.

\section{Your Assignment}

We will work with the Tecott Lab to understand the problem and then to refine
the project scope and focus.  In order to manage all of your contributions, I
will require that (1) all contributions undergo a rigorous review process, (2)
all code is tested and that those tests are included as part of our automated
test suite, and (3) all code and text components follow the best practices
adopted by many open source scientific Python packages
\cite{millman2014developing}.  All project code will be distributed under the
\href{https://en.wikipedia.org/wiki/BSD_licenses\#2-clause_license_.28.22Simplified_BSD_License.22_or_.22FreeBSD_License.22.29}{simplified
BSD license} and test data will be made publicly available.  Project
documentation will be published using \href{https://pages.github.com/}{GitHub
Pages}.

\subsection{What's different about this assignment?}

Class assignments commonly have well-defined problems to solve.  When those
assignments involve data, the typical situation is that the data are clean and
trivial to work with.  This assignment is intended to closely mimic an
applied statistics project in the real world.  A large part of the work will be
understanding the nature of problem, and making that understanding clear in
discussion, documentation, and code.  It will be your responsibility to
convince your classmates, instructors, and project sponsors that you have done
so.  The nature of the work is open-ended and uncertain.  If you haven't done
this kind of work before, it may make you uncomfortable.  But if you persist,
this will be excellent preparation for your future working life.

Given the unusual nature of the assignment, it will be instructive to distinguish
the following stages of the project:

\begin{enumerate}

\item \textbf{Understand the fundamental problem.}

This is the starting point for the project and is the unifying thread that will
run throughout all the elements of the project.  The problem provides the
context for why you are doing what you are doing.  It leads to the research questions
and hypotheses as well as guides the conclusions you make from the results of
your analysis.

\item \textbf{Data acquisition, pre-processing, and cleaning.}

As applied statisticians, we would ideally help in the experimental design part
of the project.  However, in this project (as in many applied statistics
projects), the experiment has already been conducted and the data collected.
So your next step, will be to understand the data that has been collected.  You
will also need to verify that the data you received is correct (as far as you
can tell) and that your understanding of what it represents is correct.

\item \textbf{Algorithm development.}

Before you can start writing code, you will need to decide what you want to do.
While you will want to consider computational efficiency and numerical
stability, your first and primary objective will be to determine whether your
approach to the problem is sensible given the problem you wish to address and
the data you have available.

\item \textbf{Code development.}

While you will start coding fairly early in the project, the majority of the
coding will occur in the second half of the project.  Perhaps unlike your
previous experience coding, the goal is not to produce a stream of
consciousness script.  Rather you will carefully engineer a well-designed,
rigorously tested, systematically documented codebase.  This will involve
design discussions, use case considerations, and a systematic code review
process.  It is likely that code will be heavily refactored and improved as
part of the project.

\item \textbf{Data analysis.}

Your training so far is likely to have been focused on the aspect of the
work.  Yet, this will be the last step and least involved aspect of this
project.  Once you've understood the problem and available data, determined and
implemented your algorithms, the final data analysis step should be more or
less straightforward.

\end{enumerate}

While I've sequentially enumerated the above stages of the project, it is
unlikely that the project will proceed in an entirely linear fashion.  Rather
you will likely proceed in accord with something more like a spiral plan.  In
the spiral plan, you would proceed in the order 1, 2, 1, 2, 3, 1, 2, 3, 4, 1,
2, 3, 4, 5.

\subsection*{Timeline and logistics}

Here is the tentative schedule:

\begin{table}[h]
\centering
\begin{tabular}{@{}l|l@{}}
\toprule
\multicolumn{1}{c|}{Monday} & \multicolumn{1}{c}{Wednesday} \\
\hline
(3/7) Project introduction     & (3/9) Git I \\
(3/14) Git II                  & (3/16) Git III \\
\emph{\hspace{12mm} Spring break}  & \emph{\hspace{12mm} Spring break}\\
(3/28) Start final project     & (3/30) TBD\\
(4/4) Project check in         & (4/6) TBD\\
(4/11) Project check in        & (4/13) TBD\\
(4/18) Presentation            & (4/20) TBD\\
(4/25) Project check in        & (4/27) TBD\\
(5/2) Project check in         & (5/4) TBD\\
\emph{\hspace{12mm} RRR week}  & \emph{\hspace{12mm} RRR week}\\
\emph{\hspace{12mm} Final week}  & \emph{\hspace{12mm} Final week}\\
%(5/9) RRR week                 & (5/11) RRR week\\
%(5/16) Final week              & (5/18) Final week\\
\bottomrule
\end{tabular}
\end{table}

For the remainder of the class we will be focused on the final project.  Every
Monday will be devoted to merging pull requests and planning the week's work.
Depending on how things progress, Wednesdays will be used for discussing
assigned readings or student presentations relating to the project.

On Monday, March 28th, I will go over the general project plan in class as well
as orient you to the project repository.  I will ask you to form 11 teams of 3
members each by Wednesday, March 30th.  When forming teams, you will need to
make sure that there aren't any major weaknesses on the team.  Each team will
need a mix of strengths including strong written and verbal communication
skills, core mathematical and statistical competency, as well as a programming
maturity.

On Wednesday, March 30th, teams will be assigned one of several tasks:  1)
project overview, 2) subproject descriptions, 3) project infrastructure, and 4)
data loaders and tests.  As a class, we will discuss the various tasks and come
to some agreement about what they entail.  The project overview and subproject
descriptions will focus on understanding the fundamental problem and will
involve writing a roughly one page description of the overall project as well
as one page descriptions of each subproject.  As you are initially trying to
understand the fundamental problem(s) addressed in this project, you may want
to focus as much on determining what parts of the problem you don't understand
as on the explaining the parts you do.  When trying to think about what you
don't understand, you should try to distill your lack of understanding in to
simple, direct questions.  During this process you may find that you come to
better understand the problem.  If not, you will at least have specific
questions to direct (in order) to your classmates, teaching staff, and problem
sponsors.

For each one page description, one or more teams will prepare a pull request by
the beginning of class on Monday, April 4th.  I've created a Git repository for
you with a directory with the following structure inside of the \texttt{doc}
directory:
\begin{verbatim}
problem/
|-- overview.md
|-- behavior.md
|-- path.md
|-- dynamics.md
|-- ultradian.md
|-- classification.md
`-- distribution.md
\end{verbatim}
The pull requests will be reviewed by the project sponsors as well as the
Johnny and me.  I will also ask each team to review another team's project
description.

I will discuss the details of the other tasks in more detail in class.  While
I've created the tasks for each team for the first week, I will rely on you
to help define the tasks moving forward.  However, I expect that each team
will prepare one or more substantial pull requests each week, which will be
ready for review before class on the following Monday.

I expect everyone to follow all aspects of the class project.  To help
meet my expectation, I will require that for the first few weeks teams
will take turns leading and reviewing the various aspects of the project.

On Monday, April 18th, you will present the initial work on the project in
class.  Our project sponsors will attend the presentation, so I expect to see
something substantial, professional, and interesting.  At a minimum, I expect
that you will have made substantial progress on understanding the problem, the
data, and the algorithms.  You should also have made progress on the
implementation.  In particular, I expect that you will have massively
refactored the code given to us by the project sponsors.  This refactoring
should include making the code more modular as well as documenting and testing
everything thoroughly.

Part of the goal of the presentation on April 18th will be to propose what each
team will focus on for the last half of the project.  While I will require
every team to work on all aspects of the project before the 18th, it will be up
to you to determine how you will divide up the work remaining in the project.
If there are several teams that want to work on one subproject, I may have to
assign final projects.  If this happens, I will tend to assign teams to their
preferred projects based on how much they've contributed during the first part
of the project.

%Based on the introductory
%presentation, project readings (see the Reference section at the end), your own
%research, and discussions with your classmates, you will prepare a problem
%statement at the beginning of the course.  The problem statement should provide
%a general background and context for the subprojects as well as detailed
%descriptions of the subprojects.  

\section{Python package}

By the end of the semester, you will have produced a Python package including
extensive and high quality online and printable documentation (you should view
the project documentation as your final report).  The entire class will be
responsible for the Python package and there will be one grade for the final
project.  While there will be one grade for the final project, I reserve the
right to fail anyone who doesn't fully engage in the work.

I will be responsible for creating the initial project infrastructure on
GitHub.  If there are code or infrastructure issues that the class can't agree
on, I will be responsible for making the final decision.  However, before I
intervene, you will need to carefully think through the issue and prepare
arguments for and against any decision you wish me to make.

While you will be responsible for the majority of design decisions, I will
require that the Python package have:

\begin{itemize}
\item an automated test suite with a reasonably high test coverage,
\item a comprehensive code review process for all contributed code (using
   GitHub's pull request mechanism and continuous integration using
   TravisCI and Coveralls), and
\item extensive, high quality documentation using Sphinx.
\end{itemize}

I will maintain the official project
repository\footnote{\url{https://github.com/berkeley-stat222/mouse-lifestyles}}
and will be the primary gatekeeper (i.e., I will be the one primarily
responsible for merging all pull requests).  However, I will require that pull
requests undergo a high level of review and scrutiny before I will consider
merging them.  As a class, we will develop a code review process, which will
include (among other things) program correctness, test coverage, code
readability, and style consistency.

%\subsection{Self evaluation}
%
%The purpose of the self evaluation is to provide you with an opportunity to
%reflect on your contributions and growth during the final project.
%Additionally, it is your opportunity to make sure I am aware of what you
%contributed to the final project.  You will also be asked to reflect on
%the overall course as well as the MA program in general.
%
%After spring break, I will provide additional details regarding the self
%evaluation.  I will also ask you to track your weekly goals and progress
%electronically.  You should use this weekly planning process to make
%sure your final self evaluation includes numerous concrete examples.
%

\section{Participation}

While the entire class will receive one grade for the final project, I will use
the participation portion (10\%) of your class grade to assess your
contribution to the project.  I will provide additional details about how this
will work, but here are a few things I will expect from everyone:

\begin{enumerate}
\item Attend class.  Several of you have had difficulties attending class.
  This is unacceptable.  Moving forward I expect everyone to attend every
  class.  I also expect that you will arrive prior to the start of class.
  Given the attendance problems in the first half of the class, I will
  start taking attendance.  I will provide additional details in class.
\item Participate in class.  Simply occupying space in the classroom is
  unacceptable.  You must pay attention to others.  In particular, this
  means that working on your computer when you should be listening to
  others or discussing things with your team is prohibited.  Participation
  is also more involved than merely facing others as they speak.  You will need
  to speak in front of the class.  You will need to volunteer for tasks.
  You will need to help refine ideas and contribute to the class' understanding. 
\item Contribute to all project aspects.  At a minimum every student will
  be expected to contribute to understanding the problem, developing code,
  writing tests, finding and fixing bugs, writing and revising documentation, and
  thinking through the general approach.
\end{enumerate}

This is an open-ended, real-life applied statistics project.  If you wait until
the last minute each week and try to do as little as possible, you will not
succeed.  As with anything worthwhile, you will have to work constantly and
thoughtfully.  You will need to be able to discard work that doesn't lead
anywhere and be open to trying new approaches throughout the project.

A core principle throughout the project is that you will be required to
convince me that what you did met my expectations.  In regard to the
participation portion of your grade, this means that you will ultimately be
responsible for convincing me that you've participated fully and productively.
While details will follow, you will need to submit a self-evaluation at the end
of the semester as well as possibly meet with me during finals week to discuss
your contributions.

Here are few things to keep in mind while working on your self evaluation:
\begin{enumerate}
\item Take responsibility for failures and shortcomings.
\item Do not over embellish.
\item Outline constraints you faced as well as reasons performance was hampered.
\item Include your weakness, but view them as opportunities for improvement.
\item Provide feedback on your experience during the project, course, and program.
\item Stay objective.
\item Demonstrate areas of growth.
\item Highlight skills acquired.
\item Include a discussion of problem-solving abilities you used during the project.
\end{enumerate}

I recommend that you keep notes of your contributions each week to help you
prepare your self-evaluation at the end of the semester.

\bibliographystyle{plain}
\bibliography{mouse}

\end{document}
